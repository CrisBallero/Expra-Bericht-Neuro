\appendix


\section{}
\begin{table}[h]
\centering
\captionsetup{justification=centering, margin = 1cm}
\begin{tabular}{l c c c c c c c c}
\hline
\textbf{Kategorie} & \multicolumn{3}{c}{\textbf{Community}} &  \multicolumn{3}{c}{\textbf{Distanz}} & \textbf{Anz. Trials} & \textbf{Vorh.}\\
& A & B & C & AB & BC & AC \\
\hline
1 & 1 & 1 & 1 & 1 & 1 & 3 & 18 & not B\\
2 & 1 & 1 & 1 & 1 & 2 & 3 & 12 & C\\
3 & 1 & 1 & 2 & 2 & 2 & 2 & 6 & C\\
4 & 1 & 1 & 2 & 1 & 2 & 1 & 12 & C\\
5 & 1 & 1 & 2 & 3 & 4 & 1 & 6 & not A\\
6 & 1 & 2 & 3 & 1 & 4 & 5 & 6 & C\\
7 & 1 & 2 & 3 & 4 & 4 & 4 & 18 & random\\
8 & 1 & 2 & 3 & 4 & 3 & 5 & 18 & A\\
\hline
\end{tabular}
\caption{Note. Anz. Trials = Anzahl der Durchgänge, Vorh. = Vorhersage, relevant für die Trefferquote}
\end{table}

\todo[inline]{BESCHREIBUNG DER TRIPLETS NOCH REIN?}


%%%%%%%%%%%%%%%%%%%%%%%%%%%%%%%%%%
\section{}
\begin{figure}[H]
  \centering
  \captionsetup[subfigure]{labelformat=empty}
  \setkeys{Gin}{width=2.5in,height=2.5cm,keepaspectratio}%
  \subfloat[Obj. 1a]{\includegraphics{Bilder/Objekt1A.png}}
  \hspace{0.5cm}
  \subfloat[Obj. 1b]{\includegraphics{Bilder/Objekt1B.png}} 
  \hspace{0.5cm}
  \subfloat[Obj. 2a]{\includegraphics{Bilder/Objekt2A.png}}
  \hspace{0.5cm}
  \subfloat[Obj. 2b]{\includegraphics{Bilder/Objekt2B.png}} \\
  \subfloat[Obj. 3a]{\includegraphics{Bilder/Objekt3A.png}}
  \hspace{0.5cm}
  \subfloat[Obj. 3b]{\includegraphics{Bilder/Objekt3B.png}}
  \hspace{0.5cm}
  \subfloat[Obj. 4a]{\includegraphics{Bilder/Objekt4A.png}}
  \hspace{0.5cm}
  \subfloat[Obj. 4b]{\includegraphics{Bilder/Objekt4B.png}}
  \caption{Note. a = originale Ausrichtung, b = rotiert}
\end{figure}

\begin{figure}[H]
  \ContinuedFloat\centering
  \captionsetup[subfigure]{labelformat=empty}
  \setkeys{Gin}{width=2.5in,height=2.5cm,keepaspectratio}%
  \subfloat[Obj. 5a]{\includegraphics{Bilder/Objekt5A.png}}
  \hspace{0.5cm}
  \subfloat[Obj. 5b]{\includegraphics{Bilder/Objekt5B.png}}
  \hspace{0.5cm}
  \subfloat[Obj. 6a]{\includegraphics{Bilder/Objekt6A.png}}
  \hspace{0.5cm}
  \subfloat[Obj. 6b]{\includegraphics{Bilder/Objekt6B.png}}\\
  \subfloat[Obj. 7a]{\includegraphics{Bilder/Objekt7A.png}}
  \hspace{0.5cm}
  \subfloat[Obj. 7b]{\includegraphics{Bilder/Objekt7B.png}}
  \hspace{0.5cm}
  \subfloat[Obj. 8a]{\includegraphics{Bilder/Objekt8A.png}}
  \hspace{0.5cm}
  \subfloat[Obj. 8b]{\includegraphics{Bilder/Objekt8B.png}}\\
  \subfloat[Obj. 9a]{\includegraphics{Bilder/Objekt9A.png}}
  \hspace{0.5cm}
  \subfloat[Obj. 9b]{\includegraphics{Bilder/Objekt9B.png}}
  \hspace{0.5cm}
  \subfloat[Obj. 10a]{\includegraphics{Bilder/Objekt10A.png}}
  \hspace{0.5cm}
  \subfloat[Obj. 10b]{\includegraphics{Bilder/Objekt10B.png}}\\
   \subfloat[Obj. 11a]{\includegraphics{Bilder/Objekt11A.png}}
  \hspace{0.5cm}
  \subfloat[Obj. 11b]{\includegraphics{Bilder/Objekt11B.png}}
  \hspace{0.5cm}
  \subfloat[Obj. 12a]{\includegraphics{Bilder/Objekt12A.png}}
  \hspace{0.5cm}
  \subfloat[Obj. 12b]{\includegraphics{Bilder/Objekt12B.png}}\\
   \subfloat[Obj. 13a]{\includegraphics{Bilder/Objekt13A.png}}
  \hspace{0.5cm}
  \subfloat[Obj. 13b]{\includegraphics{Bilder/Objekt13B.png}}
  \hspace{0.5cm}
  \subfloat[Obj. 14a]{\includegraphics{Bilder/Objekt14A.png}}
  \hspace{0.5cm}
  \subfloat[Obj. 14b]{\includegraphics{Bilder/Objekt14B.png}}\\
   \subfloat[Obj. 15a]{\includegraphics{Bilder/Objekt15A.png}}
  \hspace{0.5cm}
  \subfloat[Obj. 15b]{\includegraphics{Bilder/Objekt15B.png}}
  \hspace{0.5cm}
  \subfloat[Obj. 16a]{\includegraphics{Bilder/Objekt16A.png}}
  \hspace{0.5cm}
  \subfloat[Obj. 16b]{\includegraphics{Bilder/Objekt16B.png}}\\
  \end{figure}
  \begin{figure}[H]
  \ContinuedFloat\centering
  \captionsetup[subfigure]{labelformat=empty}
  \setkeys{Gin}{width=2.5in,height=2.5cm,keepaspectratio}%
   \subfloat[Obj. 17a]{\includegraphics{Bilder/Objekt17A.png}}
  \hspace{0.5cm}
  \subfloat[Obj. 17b]{\includegraphics{Bilder/Objekt17B.png}}
  \hspace{0.5cm}
  \subfloat[Obj. 18a]{\includegraphics{Bilder/Objekt18A.png}}
  \hspace{0.5cm}
  \subfloat[Obj. 18b]{\includegraphics{Bilder/Objekt18B.png}}\\
  \caption{Note. a = originale Ausrichtung, b = rotiert}
\end{figure}