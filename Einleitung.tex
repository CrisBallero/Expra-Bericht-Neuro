\textcolor{blue}{Hier noch coolen Alltagsbezug finden!}
Nach \cite{Kurby2008} nehmen wir die Welt nicht als ununterbrochenen Strom wahr, sondern unterteilen sie in koherente Ereignisse, die wiederum aus einzelnen Sequenzen bestehen. Die Event-Segmentation-Theorie von \cite{Zacks2007} besagt, dass diese Einteilung erfolgt, um kommende Information zu antizipieren.
Was dabei zusammengehört und was nicht, kategorisieren wir für semantische Inhalte nach \cite{Medin1978} aufgrund von geteilten Attributen wie z.B. Form, Verhalten und Funktion. Ereignisse können auf ähnliche Art eingeordnet werden: aufgrund ihrer zeitlichen Nähe zueinander \citep{Schapiro2013}. Die Zuordnung und Unterteilung von Ereignissen erfolgt dabei automatisch und ohne, dass wir uns darüber bewusst sind \citep{Kurby2008}, es findet also implizites Lernen statt. \cite{Berry1993} beschreibt implizites Lernen als den unbeabsichtigten Erwerb neuer Informationen, die später nur schwer verbalisierbar sind und sich intuitiv im Verhalten wiederspiegeln \citep{Kamiya2015}.

\paragraph{Organisation im Gehirn}
Damit wir (implizit) neu Erlerntes später abrufen können, müssen die Informationen und Assoziationen in einer Struktur organisiert werden \citep{Garvert2017}.
Die Organisation von zeitgleich auftretenden Stimuli wird dabei durch den Hippocampus unterstützt \citep{Schapiro2013}.
Sowohl im Hippocampus als auch im angrenzenden entorhinalen Cortex werden Assoziationen und Beziehungen zwischen Objekten, sowie deren Position im Raum reflektiert \citep{Horner2015, Schapiro2013}.
Wie \cite{Garvert2017} zeigen konnten, werden dazu unbewusst mentale Karten angelegt. In ihnen werden räumliche und abstraktere, nicht-räumliche Beziehungen wie z.B. die von Ereignissen, sich sich oft zeitlich nah ereignen, abgebildet.
Diese mentale Repräsentation von Zusammenhängen hilft den Autoren zufolge dabei, unser Verhalten zu lenken und aus bereits Gelerntem Interferenzen zu ziehen.

\paragraph{Gedächtnissysteme}
Von impliziten bzw. nicht-deklarativen Gedächtnisinhalten und ihrem Erwerb ist das explizite bzw. deklarative Gedächtnis abzugrenzen \citep{Squire1996}, dessen Inhalte bewusst reproduziert werden können. Die vorliegende Studie befasst sich mit dem Erlernen impliziter Strukturen, für deren Erwerb und Abruf weniger Aufmerksamkeit benötigt wird \citep[see][]{Gruber}.

%Dadurch, dass der Erwerb und der Abruf hierbei implizit und automatisch erfolgt, werden weniger Aufmerksamkeitsressourcen als beim Erwerb und Abruf von deklarativen Gedächtnisinhalten benötigt (Gruber, 2018)

%%%%%%%%%%%% VON JU %%%%%%%%%%%%%%%
%Erstaunlicherweise verbringt der Mensch ungefähr ein Drittel seines gesamten Lebens im Zustand des Schlafes (Terrence \& Destexhe, 2000). Es gibt viele verschiedene Funktionen des Schlafes, wovon eine sehr wahrscheinlich die Konsolidierung von neuen Gedächtnisinhalten darstellt. 
%Die Konsolidierung bezeichnet den Vorgang, indem zunächst instabile Gedächtnisspuren, in stabilere verfestigt werden (Diekelmann \& Born, 2010). Dadurch können neu enkodierte Informationen in das Langzeitgedächtnis überführt werden (Diekelmann \& Born, 2010). 

%Verschiedene Untersuchungen konnten hier einen positiven Einfluss von Schlaf auf die Konsolidierung von deklarativen, also bewussten sowie nicht-deklarativen, also unbewussten Gedächtnisinhalten nachweisen (Chambers, 2017; Rasch \& Born, 2013; Born, Rasch, \& Gais, 2006). 
%Es bleibt jedoch noch ungeklärt, ob sich ein Schlafzyklus auch positiv auf die Konsolidierung des impliziten Erwerbs von neuen deklarativen Gedächtnisinhalten und somit auch positiv auf die Erinnerungsleistung auswirkt. Daher ist das Ziel der vorliegenden Arbeit, einen möglichen moderierenden Einfluss von Schlaf auf die Interaktion zwischen diesen beiden Gedächtnissystemen zu belegen.
