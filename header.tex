\documentclass[12pt]{elsarticle}

\usepackage[]{comment}
%Seitenstruktur
\usepackage[showframe=false]{geometry}
	\geometry{a4paper,width=150mm,top=25mm,bottom=25mm,headsep=.7cm, footskip=30pt, headheight=43pt}
	%\usepackage[a4paper,width=150mm,top=25mm,bottom=25mm,bindingoffset=6mm]{geometry}
	%\geometry{a4paper,tmargin=2.2cm,bmargin=1.5cm,lmargin=\value{vlmargin}mm,rmargin=\value{vrmargin}mm, headsep=.7cm, footskip=10pt, headheight=43pt}
	
\usepackage[utf8]{inputenc} %input in ISO 8859-1 (Latin1)
\usepackage{graphicx}
\usepackage{tikz}

\usepackage{fancyhdr} % für fancy page style, inkl. footer und header
\pagestyle{fancy} 
%\newcounter{vlmargin}
%\newcounter{vrmargin}
%\setcounter{vlmargin}{18}
%\setcounter{vrmargin}{17}

\renewcommand{\footrulewidth}{.5pt}

    \fancyhf{}
    \fancyhead[RO,LE]{\itshape \leftmark}
    \fancyfoot[C]{\thepage}

\fancypagestyle{plain}{ % für seiten mit chapter titeln
    \fancyhf{}
    \fancyhead[RO,LE]{\itshape \leftmark}
%    \fancyfoot[C]{\thepage}%
     \fancyfoot[C]{\thepage}
}

%Überschriften und Kapitel
\usepackage[pagestyles]{titlesec} % custom title und section formatierung
\titleformat{\chapter}
 {\normalfont\bfseries\Huge}{\thechapter.}{10pt}{}
\titlespacing*{\chapter}{0pt}{-10pt}{20pt}

%\titleformat{\section}[display]
% {\bfseries\huge}
% {\thesection. \filright #1  }
% {-2ex} %Einstellen Abstand Chapter zu Text
 {}
%\titlespacing{\section}{0cm}{\dimexpr\topskip-10mm}{*6}[0pt]

%Sprache und Schrift
\usepackage[english]{babel}
\usepackage[utf8]{inputenc}
\usepackage[T1]{fontenc}
\usepackage{lmodern}
\usepackage{textcomp}

%Bibliothek
\usepackage[sortcites, backend=biber,
    				style=numeric, 
    				sorting=none, 
    				natbib=true, 
   					doi=false, 
                 	eprint = false,
   					url = true,
                    isbn = false,
]{biblatex}

\renewbibmacro{in:}{} % no In:
\renewcommand{\labelnamepunct}{\addcolon\space} % Doppelpunkte zwischen Author und Titel
\DeclareFieldFormat{title}{\mkbibemph{#1\isdot}}
\DeclareFieldFormat[article]{title}{#1} % keine Anführungszeichen bei den Titeln
\DeclareFieldFormat[article]{pages}{#1} % keine pages
\DeclareFieldFormat{title}{\textit{#1}} % title in italics
\renewcommand{\bibopenparen}{\addcomma\addspace}
\renewcommand{\bibcloseparen}{\addspace}

%\bibliographystyle{abbrv}
\addbibresource{ref.bib}

\usepackage[nottoc]{tocbibind} %Includes "References" in the table of contents

%\usepackage[usenames,dvipsnames]{color}

%Abbildungen und Bilder
%\usepackage{subfigure}
\usepackage{wrapfig}
\usepackage{float}
\usepackage{subcaption}
\usepackage{graphicx}
\usepackage{capt-of}

%Formeln
\usepackage{amsmath}
\usepackage{amssymb}
\usepackage{amsfonts}
\usepackage{amsthm}
\usepackage{array}
\usepackage{amstext}


%Zitate und Verweise
\usepackage{hyperref}
\usepackage{csquotes}
\usepackage{hyperref} % für links im inhaltsverzeichnis und bei \ref commands
\hypersetup{
    colorlinks=false, % bei true wird die font farbe geändert, nicht hübsch beim drucken
    citecolor=blue,
    filecolor=red,
    linkcolor=black,
    urlcolor=cyan
}
\usepackage{tabularx} % tabularx mit automatischer Spaltenbreite
\usepackage{booktabs} % tabellenverbesserungen
\usepackage{ragged2e, eqparbox} % für text-alignment mit \flushleft \flushright \centering etc

\newcolumntype{M}[1]{>{\RaggedRight}m{#1} <{\hspace*{-1pt}}}
\renewcommand{\tabularxcolumn}[1]{>{\centering\arraybackslash}m{#1}}

%\usepackage[table]{xcolor} % für tabellenzeilen hintergrundfarben
% \definecolor{Gray}{rgb}{0.80784, 0.86667, 0.90196} %dunkelblau
% \definecolor{Lightgray}{rgb}{0.9176, 0.95, 0.95686} %hellblau
% \definecolor{Akzent}{rgb}{0.6627, 0.63529, 0.55294} %akzentfarbe
%\newcommand{\whitehead}{\bfseries\color{white}}

\newcommand\colourpadding[1]{\addlinespace[-1pt]\arrayrulecolor{#1}\midrule[6pt]\arrayrulecolor{black}}


\usepackage{makecell} % für \makecell command zum besseren text formatieren in tabellenzellen
\usepackage{multirow} % multirow
\usepackage{rotating} % zum text rotieren
\usepackage{listings} % zum darstellen von code, funktioniert wie \begin{verbatim}, unterstützt syntax highlighting für mehrere programmiersprachen
\usepackage{fancyvrb} % boxen

\newcolumntype{S}[1]{>{\raggedright\arraybackslash}p{#1}}
\newcolumntype{T}[1]{>{\raggedleft\arraybackslash}p{#1}}
\newcolumntype{a}{>{\columncolor{Gray}}l}
\newcolumntype{b}{>{\columncolor{white}}l}
\newcolumntype{R}{>{\raggedright\arraybackslash}X}

\setlength\arrayrulewidth{.7pt} % müssen dicker sein als normal weil sonst das coloring die linien überdeckt




\definecolor{mygreen}{rgb}{0,0.6,0}
\definecolor{mygray}{rgb}{0.5,0.5,0.5}
\definecolor{mymauve}{rgb}{0.58,0,0.82}

\lstset{ % definiert sachen für das listings package
    language     = [LaTeX]TeX,
    delim        = [s][\color{purple}]{$}{$},
    moredelim    = [s][\color{purple}]{\\[}{\\]},
    moredelim    = [s][\color{purple}]{$$}{$$},
    moredelim    = [is][\slshape\color{violet}]{@@}{@@},  
  backgroundcolor=\color{white},   % choose the background color
  basicstyle=\footnotesize,        % size of fonts used for the code
  breaklines=true,                 % automatic line breaking only at whitespace
  captionpos=b,                    % sets the caption-position to bottom
  commentstyle=\color{mygreen},    % comment style
  escapeinside={\%*}{*)},          % if you want to add LaTeX within your code
  keywordstyle=\color{blue},       % keyword style
  stringstyle=\color{mygreen},     % string literal style
  morekeywords={*,rowcolors,multirow,dimexpr,makecell,newcolumntype,arraybackslash},
}




 %Abstände der Absätze ändern
\setlength{\parindent}{0em}
\setlength{\parskip}{0.5em}

\usepackage{multicol}
\usepackage{wrapfig}
\usepackage[export]{adjustbox}


\usepackage{chngcntr}
\counterwithin{figure}{section}

\usepackage{amsmath}

\usepackage{setspace} %Zeilenabstand ändern

\usepackage{icomma} % Leerzeile hinter Komma in Gleichung verschwindet



\usepackage[]{todonotes}

%SI units
\usepackage[separate-uncertainty = true,multi-part-units=single]{siunitx}
\usepackage{cancel}
%\usepackage{caption}
\usepackage{cleveref}
\usepackage{colortbl}
\usepackage{csquotes}
\usepackage{helvet}
\usepackage{mathpazo}
\usepackage{pgfplots}
\usepackage{xcolor}
\DeclareSIUnit\Celcius{C}
\DeclareSIUnit\pixel{pixel}

%New commands
\newcommand\tab[1][1cm]{\hspace*{#1}}

%\usepackage{emptypage}
